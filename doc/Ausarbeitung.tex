\include{Einstellungen}
% Bindet die Literaturdaten ein!
\bibliography{Bibliographie}
\begin{document}

%Startstruktur
\setcounter{secnumdepth}{3}
\setcounter{tocdepth}{2}

\pagestyle{empty}
\input{Deckblatt}

\tableofcontents
\setcounter{page}{1}

\pagestyle{scrheadings}

\newpage

\section{Flussdiagramm}

Anhand des Flussdiagramms beschreiben wir die einzelnen Regeln, die Anwendung finden.

\subsection{Gesamter Prozess}

Wir haben den gesamten Prozess in drei Teilprozesse unterteilt:
\begin{enumerate}
	\item Fahranfängerprüfung
	\item Verfügbarkeitsprüfung
	\item Preisberechnung
\end{enumerate}

Nach Aussage 3 wird zuerst geprüft, ob die Anfrage die maximale Dauer überschreitet.

\begin{figure}[tbh]
\centering
\includegraphics[width=1.0\linewidth]{Bilder/Preis_berechnen}
\caption{Preisfindung}
\label{fig:Preis_berechnen}
\end{figure}

\subsection{Fahranfängerprüfung}

In Aussage 10 wird definiert, wer als Fahranfänger eingestuft wird, und welcher Aufschlag in dem 
Fall zu zahlen ist. Aussage 11 bestimmt die Fahrzeugklassen, die nur mit Genehmigung vom Filialleiter
durch Fahranfängern gemietet werden dürfen.

\begin{figure}[H]
\centering
\includegraphics[width=0.8\linewidth]{Bilder/Fahranfaenger_pruefen}
\caption{Fahranfängerprüfung}
\label{fig:Fahranfänger_prüfen}
\end{figure}

\subsection{Verfügbarkeitsprüfung}

Aussagen: 12 (Kostenloses Upgrade nach Genehmigung), 13 (Downgrade aus Oberklasse)

In Aussage 12 wird beschrieben, was passieren soll, wenn kein Fahrzeug der angefragten Klasse
verfügbar ist. Dann wird nach Genehmigung durch den Filialleiter ein kostenloses Upgrade 
auf die nächsthöhere Klasse durchgeführt. 
Aussage 13 erläutert den Fall, wenn kein Oberklassewagen verfügbar ist. In dem Fall wird
ein Downgrade auf die Mittelklasse durchgeführt mit zusätzlichem Nachlass auf den Endpreis.

\begin{figure}[H]
\centering
\includegraphics[width=0.55\linewidth]{Bilder/Verfuegbarkeit_pruefen}
\caption{Verfügbarkeitsprüfung}
\label{fig:Verfuegbarkeit_pruefen}
\end{figure}

\subsection{Preisberechnung}

Die Preisberechnung besteht aus drei Schritten:
\begin{enumerate}
	\item Basispreisberechnung
	\item Rabattberechnung
	\item Automatikgebühr (nach Aussage 8)
\end{enumerate}

Der Endpreis setzt sich aus dem Basispreis abzüglich des Rabatts und zuzüglich 
der Automatikgebühr zusammen.

\begin{figure}[H]
	\centering
	\includegraphics[width=0.4\linewidth]{Bilder/Endpreis_berechnen}
	\caption{Endpreis berechnen}
	\label{fig:Endpreis_berechnen}
\end{figure}

\newpage 
\subsubsection{Basispreis}

Der Basispreis entspricht der Summe aller Tagespreise.

\begin{figure}[H]
	\centering
	\includegraphics[width=1.0\linewidth]{Bilder/Basispreis_berechnen}
	\caption{Basispreis berechnen}
	\label{fig:Basispreis_berechnen}
\end{figure}

\subsubsection{Tagespreis}

Bei der Tagespreisberechnung werden die Aussagen 1, 2 und 3 berücksichtigt.

Aussage 1 bestimmt den Tagespreis in Abhängigkeit von der Fahrzeugklasse.
Der Tagespreis wird allerdings an Wochenenden und Feiertagen nach Aussage 2
reduziert. Außerdem ist jeder siebte Tag nach Aussage 3 kostenlos.

\begin{figure}[H]
	\centering
	\includegraphics[width=0.9\linewidth]{Bilder/Tagespreis_berechnen}
	\caption{Tagespreis berechnen}
	\label{fig:Tagespreis_berechnen}
\end{figure}

\subsubsection{Rabatt}

Bei der Rabattberechnung werden laut Aussagen 4, 5 und 6 die Eigenschaften Neukunde, alte Reklamation
und Sicherheitstraining berücksichtigt. Dabei ist zu beachten, dass bei mehreren Fahrern die
schlechteren Konditionen gelten. Das haben wir über die "`Alle Fahrer"'-Bedingungen abgebildet (Aussage 7). Außerdem darf nach Aussage 9 der Rabatt eine bestimmte Grenze nicht überschreiten.

\begin{figure}[H]
	\centering
	\includegraphics[width=0.8\linewidth]{Bilder/Rabatt_berechnen}
	\caption{Rabatt berechnen}
	\label{fig:Rabatt_berechnen}
\end{figure}

\section{Regeln und Prozesse}

Aus den Aussagen und dem Flussdiagramm ergibt sich folgender grober Prozessablauf:
\begin{enumerate}
	\item Fahranfängerprüfung
	\item Ggf. Genehmigung der Fahrzeugklasse durch den Filialleiter
	\item Verfügbarkeitsprüfung
	\item Ggf. Genehmigung des Upgrades durch den Filialleiter
	\item Preisberechnung
\end{enumerate}

Die Genehmigungen sind manuelle Prozesse, die durch den Filialleiter durchgeführt werden müssen.
Die Fahranfängerprüfung, die Verfügbarkeitsprüfung sowie die Preisberechnung lassen sich über
Geschäftsregeln abbilden. 
Im Folgenden beschreiben wir die Regeln, die wir aus den Aussagen der Aufgabenstellung
extrahiert haben.

\subsection{Fahranfängerprüfung}

Regel: Definition eines Fahranfängers \\
Kategorie: Definitionsregel

\begin{lstlisting}
rule "Is driving novice" ruleflow-group "novice-check"
	when
		there is a request
		there is a customer with that request
		- age is less than 21 or has driving license for less than 2	
	then
		the customer is a driving novice	
end
\end{lstlisting}

Regel: Zuschlag für Fahranfänger \\
Kategorie: Prozessregel

\begin{lstlisting}
rule "Extra charge if at least one customer is a novice" ruleflow-group "novice-check"
	when
		there is a request
		there exists a customer who is driving novice
	then
		treat the request as novice
		set extra charge to 10%	
end
\end{lstlisting}

Regel: Genehmigung für höhere Klassen bei Fahranfängern \\
Kategorie: Prozessregel

\begin{lstlisting}
rule "Permission for higher classes required" ruleflow-group "novice-check"
	when
		there is a request
		- car class is not in ("Small", "Compact")
		there exists a customer who is driving novice
	then
		ask the boss for permission
end
\end{lstlisting}

\subsection{Verfügbarkeitsprüfung}

Regel: Downgrade aus Oberklasse, wenn nicht verfügbar \\
Kategorie: Prozessregel

\begin{lstlisting}
rule "Downgrade for upper class if not available" ruleflow-group "availibility-check"
	when
		there is a request
		- car class is "Upper"
		- car is not available
	then
		set car class to "Middle"
		set extra deduction to 10%
		update the request
end
\end{lstlisting}

Regel: Definition eines Upgrades für die Fahrzeugklasse \\
Kategorie: Definitionsregel

Es gibt eine Reihe von Regeln, die definieren welche Fahrzeugklassen ein Upgrade
für die angefragte Klasse darstellen. Folgende Regel ist ein Beispiel von
Klein- nach Kompaktwagen.

\begin{lstlisting}
rule "Upgrade from Small to Compact" ruleflow-group "availibility-check"
	when
		there is a request
		- car class is "Small"
		- "Compact" car is available
		there exists no upgrade
	then
		upgrade to "Compact" is possible	
end
\end{lstlisting}

Regel: Upgrade auf nächsthöhere Klasse, wenn nicht verfügbar \\
Kategorie: Prozessregel

\begin{lstlisting}
rule "If updgrade possible it requires permission" ruleflow-group "availibility-check"
	when
		there is a request
		- car is not available
		there is a possible upgrade
	then
		upgrade the car class
		mark the car for that request as available
end
\end{lstlisting}

\subsection{Preisberechnung}

Regel: Definition des Tagespreises \\
Kategorie: Definitionsregel

Es gibt eine Reihe von Regeln, die den Tagespreis für jede Fahrzeugklasse festlegen.
Folgende Regel ist ein Beispiel für Kleinwagen.

\begin{lstlisting}
rule "Daily price: Small class car" ruleflow-group "price-calculation"
	when
		there is a request
		- car class is "Small"
		there is a rental day from that request
	then
		set the daily price to 40.00
end
\end{lstlisting}

Regel: Wochenenden und Feiertage haben reduzierten Tagespreis \\
Kategorie: Ableitungsregel

\begin{lstlisting}
rule "Daily price: Weekend discount" ruleflow-group "price-calculation"
	when
		there is a request
		there is a rental day from that request
		- is weekend or holiday
	then
		the daily price is discounted by 25%
end
\end{lstlisting}

Regel: Jeder siebte Tag ist kostenlos \\
Kategorie: Ableitungsregel

\begin{lstlisting}
rule "Daily price: Free seventh day" ruleflow-group "price-calculation"
	when
		there is a request
		there is a rental day from that request
		- day index is in (7, 14, 21, 28)
	then
		set the daily price to 0.00
end
\end{lstlisting}

Regel: Basispreis ist die Summe der Tagespreise \\
Kategorie: Ableitungsregel

\begin{lstlisting}
rule "Base price is the sum of daily prices" ruleflow-group "price-calculation"
	when
		there is a request
		there is a rental day from that request
	then
		add the daily price to the base price
end
\end{lstlisting}

Regel: Neukundenrabatt \\
Kategorie: Ableitungsregel

\begin{lstlisting}
rule "Discount: All customers are new customers" ruleflow-group "price-calculation"
	when
		there is a request
		every customer from that request is a new customer
	then
		add 10.00 to the discount
end
\end{lstlisting}

Regel: Reklamationsrabatt \\
Kategorie: Ableitungsregel

\begin{lstlisting}
rule "Discount: All customers have an old reclamation" ruleflow-group "price-calculation"
	when
		there is a request
		every customer from that request had a valid reclamation
	then
		discount the base price by 20%
end
\end{lstlisting}

Regel: Rabatt für Sicherheitstraining \\
Kategorie: Ableitungsregel

\begin{lstlisting}
rule "Discount: All customers have participated in security training" ruleflow-group "price-calculation"
	when
		there is a request
		every customer from that request had a security training
	then
		discount the base price by 5%;
end
\end{lstlisting}

Regel: Maximaler Rabatt \\
Kategorie: Einschränkungsregel

\begin{lstlisting}
rule "Discount is limited to 100 Euro" ruleflow-group "price-calculation"
	when
		there is a request
		- discount is greater than 100.00
	then
		set the discount to 100.00
end
\end{lstlisting}

Regel: Endpreis berechnen \\
Kategorie: Ableitungsregel

\begin{lstlisting}
rule "Final Price: Calculation" ruleflow-group "price-calculation"
	when
		there is a request
	then
		set the final price to the base price
		subtract the discount from the final price
end
\end{lstlisting}

Regel: Automatikgebühr abziehen \\
Kategorie: Ableitungsregel

\begin{lstlisting}
rule "Final Price: Automatic fee" ruleflow-group "price-calculation"
	when
		there is a request
		- for an automatic car
	then
		add a 5% fee of the base price to the final price
end
\end{lstlisting}

Regel: Gesamtpreis berechnen \\
Kategorie: Ableitungsregel

\begin{lstlisting}
rule "Total Price" ruleflow-group "price-calculation"
	when
		there is a request
	then
		set percent to 100% plus extra charge minus extra deduction
		set the total price to percent% of the final price
end
\end{lstlisting}

\section{Anwendungssituation}

Wir analysieren die Situation am 21. März 2016 in der Autovermietungsstation.
Dazu beschreiben wir zunächst die Ausgangssituation und dann die einzelnen Kundenanfragen.

\subsection{Ausgangssituation}

Am 21. März 2016 stehen zu jeder Fahrzeugklasse zwei Fahrzeuge zur Verfügung:

\begin{tabular}{|c|c|c|c|}
	\hline \textbf{Kleinklasse} & \textbf{Kompaktklasse} & \textbf{Mittelklasse} & \textbf{Oberklasse}  \\ 
	\hline 2 & 2 & 2 & 2 \\ 
	\hline 
\end{tabular} 

Relevante Feiertage:
\begin{itemize}
	\item Fr 25.03.2016: Karfreitag
	\item Mo 28.03.2016: Ostermontag
	\item So 01.05.2016: 1. Mai / Tag der Arbeit
\end{itemize}

Vorgehen:
\begin{itemize}
	\item Wir runden mathematisch, weil das einfacher zu implementieren ist.
	\item Nicht spezifizierte Anfrageparameter nehmen einen Standardwert an (Nein, bzw. 0).
	\item Bei nicht angegebenem Alter ist der Kunde älter als 21 Jahre.
	\item Der Filialleiter genehmigt alle Anfragen.
\end{itemize}

\subsection{Kunde A}

Kundendaten:\\
\begin{tabular}{|c|c|c|c|c|}
	\hline \textbf{Alter} & \textbf{Führerschein} & \textbf{Neukunde} & \textbf{Reklamation} & \textbf{Sicherheitstraining} \\ 
	\hline 42 & 20 & Nein & Nein & Ja \\ 
	\hline 
\end{tabular} 
\\\\
Anfrage:\\
\begin{tabular}{|c|c|c|}
	\hline \textbf{Klasse} & \textbf{Dauer in Tagen} & \textbf{Automatik} \\ 
	\hline Mittel & 5 & Nein \\ 
	\hline 
\end{tabular}

\vspace{12pt}
Preisberechnung:
\begin{itemize}
	\item Zeitraum: 4 Wochentage, 0 Wochenende, 1 Feiertag
	\item Fahranfängerprüfung: Bestanden, Zuschlag: 0\%
	\item Verfügbarkeitsprüfung: Bestanden, Nachlass: 0\%
	\item Tagespreis: 70€/Wochentag, 52,50€/Feiertag
	\item Basispreis = 70€ * 4 + 52,50€ * 1 = 332,50€
	\item Rabatt = 5\% * Basispreis = 16,63€
	\item Endpreis = 315,87€
	\item Gesamtpreis = Endpreis = 315,87€
\end{itemize}

Zustand nach Vermietung:\\
\begin{tabular}{|c|c|c|c|}
	\hline \textbf{Kleinklasse} & \textbf{Kompaktklasse} & \textbf{Mittelklasse} & \textbf{Oberklasse}  \\ 
	\hline 2 & 2 & 1 & 2 \\ 
	\hline 
\end{tabular} 

\subsection{Kunde B}

Kundendaten:\\
\begin{tabular}{|c|c|c|c|c|}
	\hline \textbf{Alter} & \textbf{Führerschein} & \textbf{Neukunde} & \textbf{Reklamation} & \textbf{Sicherheitstraining} \\ 
	\hline 20 & 3 & Ja & Nein & Nein \\ 
	\hline 
\end{tabular} 
\\\\
Anfrage:\\
\begin{tabular}{|c|c|c|}
	\hline \textbf{Klasse} & \textbf{Dauer in Tagen} & \textbf{Automatik} \\ 
	\hline Mittel & 1 & Nein \\ 
	\hline 
\end{tabular}

Preisberechnung:
\begin{itemize}
	\item Zeitraum: 1 Wochentag, 0 Wochenende oder Feiertage
	\item Fahranfängerprüfung: Genehmigung, Bestanden, Zuschlag: 10\%
	\item Verfügbarkeitsprüfung: Bestanden, Nachlass: 0\%
	\item Tagespreis: 70€/Wochentag, 52,50€/Feiertag
	\item Basispreis = 70€ * 1 + 52,50€ * 0 = 70,00€
	\item Rabatt = 10,00€
	\item Endpreis = 60,00€
	\item Gesamtpreis = 110\% * Endpreis = 66,00€
\end{itemize}

Zustand nach Vermietung:\\
\begin{tabular}{|c|c|c|c|}
	\hline \textbf{Kleinklasse} & \textbf{Kompaktklasse} & \textbf{Mittelklasse} & \textbf{Oberklasse}  \\ 
	\hline 2 & 2 & 0 & 2 \\ 
	\hline 
\end{tabular} 

\subsection{Kunde C}

Kundendaten:\\
\begin{tabular}{|c|c|c|c|c|}
	\hline \textbf{Alter} & \textbf{Führerschein} & \textbf{Neukunde} & \textbf{Reklamation} & \textbf{Sicherheitstraining} \\ 
	\hline 30 & 12 & Nein & Nein & Nein \\ 
	\hline 
\end{tabular} 
\\\\
Anfrage:\\
\begin{tabular}{|c|c|c|}
	\hline \textbf{Klasse} & \textbf{Dauer in Tagen} & \textbf{Automatik} \\ 
	\hline Mittel & 7 & Nein \\ 
	\hline 
\end{tabular}

Zwischenergebnis:
\begin{itemize}
	\item Zeitraum: 4 Wochentag, 2 Wochenende oder Feiertage, 1 Gratistag
	\item Fahranfängerprüfung: Bestanden, Zuschlag: 0\%
	\item Verfügbarkeitsprüfung: Genehmigung, Bestanden, Upgrade auf höhere Klasse, Nachlass: 0\%
	\item Tagespreis: 70€/Wochentag, 52,50€/Feiertag
	\item Basispreis = 70€ * 4 + 52,50€ * 2 + 0€ * 1 = 385,00€
	\item Rabatt = 0,00€
	\item Endpreis = 385,00€
	\item Gesamtpreis = 100\% * Endpreis = 385,00€
\end{itemize}

Zustand nach Vermietung:\\
\begin{tabular}{|c|c|c|c|}
	\hline \textbf{Kleinklasse} & \textbf{Kompaktklasse} & \textbf{Mittelklasse} & \textbf{Oberklasse}  \\ 
	\hline 2 & 2 & 0 & 1 \\ 
	\hline 
\end{tabular} 

\subsection{Kunde D}

Kundendaten:\\
\begin{tabular}{|c|c|c|c|c|}
	\hline \textbf{Alter} & \textbf{Führerschein} & \textbf{Neukunde} & \textbf{Reklamation} & \textbf{Sicherheitstraining} \\ 
	\hline 35 & 10 & Ja & Nein & Nein \\ 
	\hline 
\end{tabular} 
\\\\
Anfrage:\\
\begin{tabular}{|c|c|c|}
	\hline \textbf{Klasse} & \textbf{Dauer in Tagen} & \textbf{Automatik} \\ 
	\hline Mittel & 1 & Nein \\ 
	\hline 
\end{tabular}

Preisberechnung:
\begin{itemize}
	\item Zeitraum: 1 Wochentag, 0 Wochenende oder Feiertage
	\item Fahranfängerprüfung: Bestanden, Zuschlag: 0\%
	\item Verfügbarkeitsprüfung: Genehmigung, Bestanden, Upgrade auf höhere Klasse, Nachlass: 0\%
	\item Tagespreis: 70€/Wochentag, 52,50€/Feiertag
	\item Basispreis = 70€ * 1 + 52,50€ * 0 = 70,00€
	\item Rabatt = 10,00€
	\item Endpreis = 60,00€
	\item Gesamtpreis = 100\% * Endpreis = 60,00€
\end{itemize}

Zustand nach Vermietung:\\
\begin{tabular}{|c|c|c|c|}
	\hline \textbf{Kleinklasse} & \textbf{Kompaktklasse} & \textbf{Mittelklasse} & \textbf{Oberklasse}  \\ 
	\hline 2 & 2 & 0 & 0 \\ 
	\hline 
\end{tabular}

\subsection{Kunde E}

Kundendaten:\\
\begin{tabular}{|c|c|c|c|c|}
	\hline \textbf{Alter} & \textbf{Führerschein} & \textbf{Neukunde} & \textbf{Reklamation} & \textbf{Sicherheitstraining} \\ 
	\hline 46 & 20 & Ja & Nein & Nein \\ 
	\hline 
\end{tabular} 
\\\\
Anfrage:\\
\begin{tabular}{|c|c|c|}
	\hline \textbf{Klasse} & \textbf{Dauer in Tagen} & \textbf{Automatik} \\ 
	\hline Ober & 6 & Nein \\ 
	\hline 
\end{tabular}

Preisberechnung:
\begin{itemize}
	\item Zeitraum: 4 Wochentag, 2 Wochenende oder Feiertage
	\item Fahranfängerprüfung: Bestanden, Zuschlag: 0\%
	\item Verfügbarkeitsprüfung: Genehmigung, Abgelehnt, Nachlass: 10\%, Klasse = Mittel
	
	% Es gibt noch eine Oberklasse:
	%\item Tagespreis: 90€/Wochentag, 67,50€/Feiertag
	%\item Basispreis = $90€ * 4 + 67,50€ * 2 = 495,00€$
%	\item Rabatt = 10,00€
%	\item Endpreis = 485,00€
%	\item Gesamtpreis = 100\% * Endpreis = 485,00€
	
	% Es gibt keine Oberklasse aber eine Mittelklasse
%	\item Tagespreis: 70€/Wochentag, 52,50€/Feiertag
%	\item Basispreis = $70€ * 4 + 52,50€ * 2 = 385,00€$
%	\item Rabatt = 10,00€
%	\item Endpreis = 375,00€
%	\item Gesamtpreis = 90\% * Endpreis = 337,50€
\end{itemize}

Abgelehnt: Keine Änderung
Zustand nach Vermietung:\\
\begin{tabular}{|c|c|c|c|}
	\hline \textbf{Kleinklasse} & \textbf{Kompaktklasse} & \textbf{Mittelklasse} & \textbf{Oberklasse}  \\ 
	\hline 2 & 2 & 0 & 0 \\ 
	\hline 
\end{tabular}

\subsection{Kunde F}

Kundendaten:\\
\begin{tabular}{|c|c|c|c|c|}
	\hline \textbf{Alter} & \textbf{Führerschein} & \textbf{Neukunde} & \textbf{Reklamation} & \textbf{Sicherheitstraining} \\ 
	\hline 32 & 10 & Nein & Nein & Ja \\ 
	\hline 
\end{tabular} 
\\\\
Anfrage:\\
\begin{tabular}{|c|c|c|}
	\hline \textbf{Klasse} & \textbf{Dauer in Tagen} & \textbf{Automatik} \\ 
	\hline Ober & 4 & Nein \\ 
	\hline 
\end{tabular}

Zwischenergebnis:
\begin{itemize}
	\item Zeitraum: 4 Wochentag, 0 Wochenende oder Feiertage
	\item Fahranfängerprüfung: Bestanden, Zuschlag: 0\%
	\item Verfügbarkeitsprüfung: Genehmigung, Abgelehnt, Nachlass: 0\%
%	\item Tagespreis: 90€/Wochentag, 67,50€/Feiertag
%	\item Basispreis = $90€ * 4 + 67,50€ * 0 = 360,00€$
%	\item Rabatt = 20\% * 360€ + 5\% * 360€ = 72€ + 18€ = 90€
%	\item Endpreis = 270,00€
%	\item Gesamtpreis = 100\% * Endpreis = 270,00€
	
	% Siehe E
\end{itemize}

Abgelehnt: Keine Änderung
Zustand nach Vermietung:\\
\begin{tabular}{|c|c|c|c|}
	\hline \textbf{Kleinklasse} & \textbf{Kompaktklasse} & \textbf{Mittelklasse} & \textbf{Oberklasse}  \\ 
	\hline 2 & 2 & 0 & 0 \\ 
	\hline 
\end{tabular}

\subsection{Kunde G}

Kundendaten:\\
\begin{tabular}{|c|c|c|c|c|}
	\hline \textbf{Alter} & \textbf{Führerschein} & \textbf{Neukunde} & \textbf{Reklamation} & \textbf{Sicherheitstraining} \\ 
	\hline 30 & 6 & Nein & Nein & Ja \\ 
	\hline 25 & 1 & Nein & Nein & Nein \\ 
	\hline 
\end{tabular} 
\\\\
Anfrage:\\
\begin{tabular}{|c|c|c|}
	\hline \textbf{Klasse} & \textbf{Dauer in Tagen} & \textbf{Automatik} \\ 
	\hline Kompakt & 1 & Nein \\ 
	\hline 
\end{tabular}

Zwischenergebnis:
\begin{itemize}
	\item Zeitraum: 1 Wochentag, 0 Wochenende oder Feiertage
	\item Fahranfängerprüfung: Bestanden, Zuschlag: 10\%
	\item Verfügbarkeitsprüfung: Bestanden, Nachlass: 0\%
	\item Tagespreis: 50€/Wochentag, 37,50€/Feiertag
	\item Basispreis = 50€ * 1 + 37,50€ * 0 = 50,00€
	\item Rabatt = 0€
	\item Endpreis = 50,00€
	\item Gesamtpreis = 110\% * Endpreis = 55,00€
\end{itemize}

Zustand nach Vermietung:\\
\begin{tabular}{|c|c|c|c|}
	\hline \textbf{Kleinklasse} & \textbf{Kompaktklasse} & \textbf{Mittelklasse} & \textbf{Oberklasse}  \\ 
	\hline 2 & 2 & 0 & 0 \\ 
	\hline 
\end{tabular}

\section{Prototyp}

Hier den Prototypen beschreiben.

%\appendix
%\include{Inhalt/Anhang}

%\newpage
%\printbibliography

\end{document}