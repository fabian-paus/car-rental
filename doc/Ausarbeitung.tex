\include{Einstellungen}
% Bindet die Literaturdaten ein!
\bibliography{Bibliographie}
\begin{document}

%Startstruktur
\setcounter{secnumdepth}{3}
\setcounter{tocdepth}{2}

\pagestyle{empty}
\input{Deckblatt}

\tableofcontents
\setcounter{page}{1}

\pagestyle{scrheadings}

\newpage

\section{Regeln und Workflows}

Hier wandeln wir die Aussagen in Regeln und Workflows.

\section{Flussdiagramm}

Hier das Flussdiagramm der kompletten Preisfindung.

\section{Einordnung der Regeln}

Die Regeln in die Regelklassen einordnen.

\section{Anwendungssituation}

Wir analysieren die Situation am 21. März 2016 in der Autovermietungsstation.
Dazu beschreiben wir zunächst die Ausgangssituation und dann die einzelnen Kundenanfragen.

\subsection{Ausgangssituation}



Am 21. März 2016 stehen zu jeder Fahrzeugklasse zwei Fahrzeuge zur Verfügung:
\begin{description}
	\item[Kleinklasse:] 2
	\item[Kompaktklasse:] 2
	\item[Mittelklasse:] 2
	\item[Oberklasse:] 2
\end{description}

\section{Prototyp}

Hier den Prototypen beschreiben.

%\appendix
%\include{Inhalt/Anhang}

%\newpage
%\printbibliography

\end{document}