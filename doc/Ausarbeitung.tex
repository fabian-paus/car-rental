\include{Einstellungen}
% Bindet die Literaturdaten ein!
\bibliography{Bibliographie}
\begin{document}

%Startstruktur
\setcounter{secnumdepth}{3}
\setcounter{tocdepth}{2}

\pagestyle{empty}
\input{Deckblatt}

\tableofcontents
\setcounter{page}{1}

\pagestyle{scrheadings}

\newpage

\section{Regeln und Workflows}

Hier wandeln wir die Aussagen in Regeln und Workflows.

\section{Flussdiagramm}

Hier das Flussdiagramm der kompletten Preisfindung.

\section{Einordnung der Regeln}

Die Regeln in die Regelklassen einordnen.

\section{Anwendungssituation}

Wir analysieren die Situation am 21. März 2016 in der Autovermietungsstation.
Dazu beschreiben wir zunächst die Ausgangssituation und dann die einzelnen Kundenanfragen.

\subsection{Ausgangssituation}

Am 21. März 2016 stehen zu jeder Fahrzeugklasse zwei Fahrzeuge zur Verfügung:
\begin{description}
	\item[Kleinklasse:] 2
	\item[Kompaktklasse:] 2
	\item[Mittelklasse:] 2
	\item[Oberklasse:] 2
\end{description}

Relevante Feiertage:
\begin{itemize}
	\item Fr 25.03.2016: Karfreitag
	\item Mo 28.03.2016: Ostermontag
	\item So 01.05.2016: 1. Mai / Tag der Arbeit
\end{itemize}

Klärung:
\begin{itemize}
	\item Wir runden mathematisch, weil das einfacher zu implementieren ist.
	\item Nicht spezifizierte Anfrageparameter nehmen einen Standardwert an (Nein, bzw. 0).
	\item Bei nicht angegebenem Alter ist der Kunde älter als 21 Jahre.
	\item Der Filialleiter genehmigt alle Anfragen.
\end{itemize}

\subsection{Kunde A}

Kundenanfrage:
\begin{itemize}
	\item Klasse: Mittel
	\item Dauer: 5 Tage
	\item Automatik: Nein
	\item Sicherheitstraining: Ja
	
	\item Alter: > 21 Jahre
	\item Neukunde: Nein
	\item Alte Reklamation: Nein
	\item Führerscheindauer: 20 Jahre
\end{itemize}

Zwischenergebnis:
\begin{itemize}
	\item Zeitraum: 4 Wochentage, 0 Wochenende, 1 Feiertag
	\item Fahranfängerprüfung: Bestanden, Zuschlag: 0\%
	\item Verfügbarkeitsprüfung: Bestanden, Nachlass: 0\%
	\item Tagespreis: 70€/Wochentag, 52,50€/Feiertag
	\item Basispreis = $70€ * 4 + 52,50€ * 1 = 332,50€$
	\item Rabatt = 5\% * Basispreis = 16,63€
	\item Endpreis = 315,87€
	\item Gesamtpreis = Endpreis = 315,87€
\end{itemize}

Zustand nach Vermietung:
\begin{description}
	\item[Kleinklasse:] 2
	\item[Kompaktklasse:] 2
	\item[Mittelklasse:] 1
	\item[Oberklasse:] 2
\end{description}

\subsection{Kunde B}

Kundenanfrage:
\begin{itemize}
	\item Klasse: Mittel
	\item Dauer: 1 Tage
	\item Automatik: Nein
	
	\item Alter: 20 Jahre
	\item Sicherheitstraining: Nein
	\item Neukunde: Ja
	\item Alte Reklamation: Nein
	\item Führerscheindauer: 3 Jahre
\end{itemize}

Zwischenergebnis:
\begin{itemize}
	\item Zeitraum: 1 Wochentag, 0 Wochenende oder Feiertage
	\item Fahranfängerprüfung: Genehmigung, Bestanden, Zuschlag: 10\%
	\item Verfügbarkeitsprüfung: Bestanden, Nachlass: 0\%
	\item Tagespreis: 70€/Wochentag, 52,50€/Feiertag
	\item Basispreis = $70€ * 1 + 52,50€ * 0 = 70,00€$
	\item Rabatt = 10,00€
	\item Endpreis = 60,00€
	\item Gesamtpreis = 110\% * Endpreis = 66,00€
\end{itemize}

Zustand nach Vermietung:
\begin{description}
	\item[Kleinklasse:] 2
	\item[Kompaktklasse:] 2
	\item[Mittelklasse:] 0
	\item[Oberklasse:] 2
\end{description}

\subsection{Kunde C}

Kundenanfrage:
\begin{itemize}
	\item Klasse: Mittel
	\item Dauer: 7 Tage
	\item Automatik: Nein
	
	\item Alter: 30 Jahre
	\item Sicherheitstraining: Nein
	\item Neukunde: Nein
	\item Alte Reklamation: Nein
	\item Führerscheindauer: 12 Jahre
\end{itemize}

Zwischenergebnis:
\begin{itemize}
	\item Zeitraum: 4 Wochentag, 2 Wochenende oder Feiertage, 1 Gratistag
	\item Fahranfängerprüfung: Bestanden, Zuschlag: 0\%
	\item Verfügbarkeitsprüfung: Genehmigung, Bestanden, Upgrade auf höhere Klasse, Nachlass: 0\%
	\item Tagespreis: 70€/Wochentag, 52,50€/Feiertag
	\item Basispreis = $70€ * 4 + 52,50€ * 2 + 0€ * 1 = 385,00€$
	\item Rabatt = 0,00€
	\item Endpreis = 385,00€
	\item Gesamtpreis = 100\% * Endpreis = 385,00€
\end{itemize}

Zustand nach Vermietung:
\begin{description}
	\item[Kleinklasse:] 2
	\item[Kompaktklasse:] 2
	\item[Mittelklasse:] 0
	\item[Oberklasse:] 1
\end{description}

\subsection{Kunde D}

Kundenanfrage:
\begin{itemize}
	\item Klasse: Mittel
	\item Dauer: 1 Tage
	\item Automatik: Nein
	
	\item Alter: 35 Jahre
	\item Sicherheitstraining: Nein
	\item Neukunde: Ja
	\item Alte Reklamation: Nein
	\item Führerscheindauer: 10 Jahre
\end{itemize}

Zwischenergebnis:
\begin{itemize}
	\item Zeitraum: 1 Wochentag, 0 Wochenende oder Feiertage
	\item Fahranfängerprüfung: Bestanden, Zuschlag: 0\%
	\item Verfügbarkeitsprüfung: Genehmigung, Bestanden, Upgrade auf höhere Klasse, Nachlass: 0\%
	\item Tagespreis: 70€/Wochentag, 52,50€/Feiertag
	\item Basispreis = $70€ * 1 + 52,50€ * 0 = 70,00€$
	\item Rabatt = 10,00€
	\item Endpreis = 60,00€
	\item Gesamtpreis = 100\% * Endpreis = 60,00€
\end{itemize}

Zustand nach Vermietung:
\begin{description}
	\item[Kleinklasse:] 2
	\item[Kompaktklasse:] 2
	\item[Mittelklasse:] 0
	\item[Oberklasse:] 0
\end{description}

\subsection{Kunde E}

Kundenanfrage:
\begin{itemize}
	\item Klasse: Ober
	\item Dauer: 6 Tage
	\item Automatik: Nein
	
	\item Alter: 46 Jahre
	\item Sicherheitstraining: Nein
	\item Neukunde: Ja
	\item Alte Reklamation: Nein
	\item Führerscheindauer: 20 Jahre
\end{itemize}

Zwischenergebnis:
\begin{itemize}
	\item Zeitraum: 4 Wochentag, 2 Wochenende oder Feiertage
	\item Fahranfängerprüfung: Bestanden, Zuschlag: 0\%
	\item Verfügbarkeitsprüfung: Genehmigung, Abgelehnt, Nachlass: 10\%, Klasse = Mittel
	
	% Es gibt noch eine Oberklasse:
	\item Tagespreis: 90€/Wochentag, 67,50€/Feiertag
	\item Basispreis = $90€ * 4 + 67,50€ * 2 = 495,00€$
	\item Rabatt = 10,00€
	\item Endpreis = 485,00€
	\item Gesamtpreis = 100\% * Endpreis = 485,00€
	
	% Es gibt keine Oberklasse aber eine Mittelklasse
	\item Tagespreis: 70€/Wochentag, 52,50€/Feiertag
	\item Basispreis = $70€ * 4 + 52,50€ * 2 = 385,00€$
	\item Rabatt = 10,00€
	\item Endpreis = 375,00€
	\item Gesamtpreis = 90\% * Endpreis = 346,50€
\end{itemize}

Abgelehnt: Keine Änderung
Zustand nach Vermietung:
\begin{description}
	\item[Kleinklasse:] 2
	\item[Kompaktklasse:] 2
	\item[Mittelklasse:] 0
	\item[Oberklasse:] 0
\end{description}

\subsection{Kunde F}

Kundenanfrage:
\begin{itemize}
	\item Klasse: Ober
	\item Dauer: 4 Tage
	\item Automatik: Nein
	
	\item Alter: 32 Jahre
	\item Sicherheitstraining: Ja
	\item Neukunde: Nein
	\item Alte Reklamation: Ja
	\item Führerscheindauer: 10 Jahre
\end{itemize}

Zwischenergebnis:
\begin{itemize}
	\item Zeitraum: 4 Wochentag, 0 Wochenende oder Feiertage
	\item Fahranfängerprüfung: Bestanden, Zuschlag: 0\%
	\item Verfügbarkeitsprüfung: Genehmigung, Abgelehnt, Nachlass: 0\%
	\item Tagespreis: 90€/Wochentag, 67,50€/Feiertag
	\item Basispreis = $90€ * 4 + 67,50€ * 0 = 360,00€$
	\item Rabatt = 20\% * 360€ + 5\% * 360€ = 72€ + 18€ = 90€
	\item Endpreis = 270,00€
	\item Gesamtpreis = 100\% * Endpreis = 270,00€
	
	% Siehe E
\end{itemize}

Abgelehnt: Keine Änderung
Zustand nach Vermietung:
\begin{description}
	\item[Kleinklasse:] 2
	\item[Kompaktklasse:] 2
	\item[Mittelklasse:] 0
	\item[Oberklasse:] 0
\end{description}

\subsection{Kunde G}

Kundenanfrage:
\begin{itemize}
	\item Klasse: Kompakt
	\item Dauer: 1 Tage
	\item Automatik: Nein
	
	\item Alter: > 21 Jahre
	\item Sicherheitstraining: Nein (wg. Frau)
	\item Neukunde: Nein
	\item Alte Reklamation: Nein
	\item Führerscheindauer: mind. 6 Jahre 
	\item Fahranfänger wg. Frau
\end{itemize}

Zwischenergebnis:
\begin{itemize}
	\item Zeitraum: 1 Wochentag, 0 Wochenende oder Feiertage
	\item Fahranfängerprüfung: Bestanden, Zuschlag: 10\%
	\item Verfügbarkeitsprüfung: Bestanden, Nachlass: 0\%
	\item Tagespreis: 50€/Wochentag, 37,50€/Feiertag
	\item Basispreis = $50€ * 1 + 37,50€ * 0 = 50,00€$
	\item Rabatt = 0€
	\item Endpreis = 50,00€
	\item Gesamtpreis = 110\% * Endpreis = 55,00€
\end{itemize}

Zustand nach Vermietung:
\begin{description}
	\item[Kleinklasse:] 2
	\item[Kompaktklasse:] 1
	\item[Mittelklasse:] 0
	\item[Oberklasse:] 0
\end{description}

\section{Prototyp}

Hier den Prototypen beschreiben.

%\appendix
%\include{Inhalt/Anhang}

%\newpage
%\printbibliography

\end{document}