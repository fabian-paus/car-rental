\include{Einstellungen}
% Bindet die Literaturdaten ein!
\bibliography{Bibliographie}
\begin{document}

%Startstruktur
\setcounter{secnumdepth}{3}
\setcounter{tocdepth}{2}

\pagestyle{empty}
\input{Deckblatt}

\tableofcontents
\setcounter{page}{1}

\pagestyle{scrheadings}

\newpage

\section{Einleitung}

Aufgabe dieses Projekts war es eine Autovermietung mithilfe von jBPM und Drools zu erstellen. Die Geschäftsregeln sollen am Ende über eine Domain-Specific-Language (DSL) verwaltet werden. Zur Definition der Regeln wurden die dreizehn Aussagen aus der Aufgabenstellung analysiert und ausgewertet. 

Im Laufe dieser Projektdokumentation werden die einzelnen Arbeitsschritte strukturiert und detailliert beschrieben. Der Quellcode und die Projektdokumentation kann über ein öffentliches Git-Repository (\url{https://github.com/fabian-paus/car-rental}) auf GitHub.com eingesehen werden. 

Als Ergebnis haben wir eine grafische Benutzeroberfläche entworfen, über die eine Miet\-anfrage an die Prozess-Engine gesendet werden kann. Die Prozess-Engine verarbeitet die Anfrage und nutzt die Regel-Engine zur Ausführung von Geschäftsregeln.
Unter gewissen Bedingungen muss der Filialleiter Genehmigungen erteilen.
Am Ende wird die Anfrage entweder angenommen oder abgelehnt. Eine angenommene Anfrage enthält dann eine Fahrzeugklasse und einen Gesamtpreis. Diese Daten werden abschließend über die Oberfläche an den Benutzer ausgegeben. 

\section{Einarbeitung}

Um die Aufgabe besser zu strukturieren, definieren wir zunächst einige Domänenbegriffe und 
beschreiben dann den Ablauf des Vermietungsprozesses mithilfe eines Flussdiagramms.

\subsection{Domänenbegriffe}

In diesem Abschnitt werden Begriffe, die für den Vermietungsprozess wichtig sind, definiert.
Diese Begriffe werden in Diagrammen, in Regeln und bei der Prozessdefinition verwendet.
Einige Begriffe werden nicht exakt definiert, da diese später in Regeln genauer
festgelegt werden.

\begin{description}
	\item[Kunde:] Ein Kunde möchte einen Wagen mieten. Für jeden Kunden sind folgende Eigenschaften bekannt:
		\begin{description}
			\item[Alter:] Wie alt ist der Kunde? Einheit: Jahre
			\item[Führerscheindauer:] Wie lange hat der Kunde seinen Führerschein? Einheit: Jahre
			\item[Sicherheitstraining:] Hat der Kunde ein Fahrsicherheitstraining absolviert?
			\item[Reklamation:] Hat der Kunde eine berechtigte Reklamation vorgebracht?
			\item[Neukunde:] Ist der Kunde zum ersten Mal bei der Autovermietung?
		\end{description}
	\item[Fahranfänger:] Ein Kunde ist ein Fahranfänger, wenn er jünger als ein bestimmtes Alter ist oder
		seinen Führerschein noch nicht sehr lange besitzt. Die genaue Definition erfolgt anhand einer
		Geschäftsregel.	
	\item[Filialleiter:] Gewisse Entscheidung beim Vermietungsprozess bedürfen der Genehmigung durch
		den Filialleiter.
	\item[Klasse:] Ein zu mietender Wagen wird in vier Klassen eingeteilt:
		\begin{description}
			\item[Klein:] Kleinwagen
			\item[Kompakt:] Kompaktwagen
			\item[Mittel:] Mittelklassewagen
			\item[Ober:] Oberklassewagen
		\end{description}
	\item[Anfrage:] Eine Anfrage für einen Mietwagen wird von einem Kunden gestellt. Sie enthält folgende Informationen:
		\begin{description}
			\item[Fahrer:] Welche Kunden fahren mit dem Auto? (Achtung: mehr als ein Fahrer möglich)
			\item[Startdatum:] Ab welchem Datum wird das Auto gemietet? Einheit: Datum
			\item[Dauer:] Für wie lange soll das Auto gemietet werden? Einheit: Tage
			\item[Klasse:] Welche Wagenklasse soll gemietet werden? Einheit: Klasse (siehe Definition oben)
			\item[Automatik:] Soll der Mietwagen eine Automatikschaltung haben?
		\end{description}
	\item[Tagespreis:] Der Tagespreis ist der Preis für eine Mietdauer von einem Tag. Einheit: Euro
	\item[Basispreis:] Der Basispreis ist die Summe der Tagespreise über die gesamte Dauer. Einheit: Euro
	\item[Rabatt:] Der Rabatt ist ein Nachlass, der auf Basis der spezifischen Kundendaten ermittelt wird. Einheit: Euro
	\item[Automatikgebühr:] Eine zusätzliche Gebühr, falls ein Automatikwagen gemietet wird. Einheit: Euro
	\item[Endpreis:] Der Endpreis wird aus dem Basispreis, dem Rabatt und der Automatikgebühr berechnet. Einheit: Euro
	\item[Zuschlag:] Ein prozentualer Zuschlag auf den Endpreis. Einheit: Prozent
	\item[Nachlass:] Ein prozentualer Nachlass auf den Endpreis. Einheit: Prozent
	\item[Gesamtpreis:] Nachdem Zuschlag und Nachlass auf den Endpreis berechnet wurden, ergibt sich der Gesamtpreis. Einheit: Euro
	
\end{description}

\subsection{Flussdiagramm}

Anhand des Flussdiagramms beschreiben wir den gesamten Vermietungsprozess. Dabei gehen wir auf
die einzelnen Aussagen aus der Aufgabenstellung ein (siehe Anhang \ref{anh:Aufgabe}).

\subsubsection{Gesamter Prozess}

Wir haben den gesamten Prozess in drei Teilprozesse unterteilt:
\begin{enumerate}
	\item Fahranfängerprüfung
	\item Verfügbarkeitsprüfung
	\item Preisberechnung
\end{enumerate}

\begin{figure}[tbh]
\centering
\includegraphics[width=1.0\linewidth]{Bilder/Preis_berechnen}
\caption{Preisfindung}
\label{fig:Preis_berechnen}
\end{figure}

\subsubsection{Fahranfängerprüfung}

In Aussage 10 wird definiert, wer als Fahranfänger eingestuft wird, und welcher Aufschlag in dem 
Fall zu zahlen ist. Aussage 11 bestimmt die Fahrzeugklassen, die nur mit Genehmigung vom Filialleiter
durch Fahranfängern gemietet werden dürfen.

\begin{figure}[H]
\centering
\includegraphics[width=1.0\linewidth]{Bilder/Fahranfaenger_pruefen}
\caption{Fahranfängerprüfung}
\label{fig:Fahranfänger_prüfen}
\end{figure}

\subsubsection{Verfügbarkeitsprüfung}

In Aussage 12 wird beschrieben, was passieren soll, wenn kein Fahrzeug der angefragten Klasse
verfügbar ist. Dann wird nach Genehmigung durch den Filialleiter ein kostenloses Upgrade 
auf die nächsthöhere Klasse durchgeführt. 
Aussage 13 erläutert den Fall, wenn kein Oberklassewagen verfügbar ist. In diesem Fall wird
ein Downgrade auf die Mittelklasse durchgeführt mit zusätzlichem Nachlass auf den Endpreis (siehe Abbildung \ref{fig:Verfuegbarkeit_pruefen}).

\begin{figure}[p]
\centering
\includegraphics[width=0.75\linewidth]{Bilder/Verfuegbarkeit_pruefen}
\caption{Verfügbarkeitsprüfung}
\label{fig:Verfuegbarkeit_pruefen}
\end{figure}

\subsubsection{Preisberechnung}

Die Preisberechnung besteht aus drei Schritten:
\begin{enumerate}
	\item Basispreisberechnung
	\item Rabattberechnung
	\item Automatikgebühr (nach Aussage 8)
\end{enumerate}

Der Endpreis setzt sich aus dem Basispreis abzüglich des Rabatts und zuzüglich 
der Automatikgebühr zusammen. Die Berechnung wird in Abbildung \ref{fig:Endpreis_berechnen}
dargestellt.

\begin{figure}[p]
	\centering
	\includegraphics[width=0.6\linewidth]{Bilder/Endpreis_berechnen}
	\caption{Endpreis berechnen}
	\label{fig:Endpreis_berechnen}
\end{figure}

\newpage 
\paragraph{Basispreis}

Der Basispreis entspricht der Summe aller Tagespreise.

\begin{figure}[H]
	\centering
	\includegraphics[width=0.8\linewidth]{Bilder/Basispreis_berechnen}
	\caption{Basispreis berechnen}
	\label{fig:Basispreis_berechnen}
\end{figure}

\paragraph{Tagespreis}

Bei der Tagespreisberechnung werden die Aussagen 1, 2 und 3 berücksichtigt.

Aussage 1 bestimmt den Tagespreis in Abhängigkeit von der Fahrzeugklasse.
Der Tagespreis wird allerdings an Wochenenden und Feiertagen nach Aussage 2
reduziert. Außerdem ist jeder siebte Tag nach Aussage 3 kostenlos (siehe Abbildung
\ref{fig:Tagespreis_berechnen}).

\begin{figure}[p]
	\centering
	\includegraphics[width=0.9\linewidth]{Bilder/Tagespreis_berechnen}
	\caption{Tagespreis berechnen}
	\label{fig:Tagespreis_berechnen}
\end{figure}

\paragraph{Rabatt}

Bei der Rabattberechnung werden laut Aussagen 4, 5 und 6 die Eigenschaften Neukunde, alte Reklamation
und Sicherheitstraining berücksichtigt. Dabei ist zu beachten, dass bei mehreren Fahrern die
schlechteren Konditionen gelten. Das haben wir über die "`Alle Fahrer"'-Bedingungen abgebildet (Aussage 7). Außerdem darf nach Aussage 9 der Rabatt eine bestimmte Grenze nicht überschreiten
(siehe Abbildung \ref{fig:Rabatt_berechnen}).

\begin{figure}[p]
	\centering
	\includegraphics[width=0.8\linewidth]{Bilder/Rabatt_berechnen}
	\caption{Rabatt berechnen}
	\label{fig:Rabatt_berechnen}
\end{figure}

\newpage
\section{Regeln und Prozesse}
\label{sec:rules}

Aus den Aussagen und dem Flussdiagramm ergibt sich folgender grober Prozessablauf:
\begin{enumerate}
	\item Fahranfängerprüfung
	\item Ggf. Genehmigung der Fahrzeugklasse durch den Filialleiter
	\item Verfügbarkeitsprüfung
	\item Ggf. Genehmigung des Upgrades durch den Filialleiter
	\item Preisberechnung
\end{enumerate}

Die Genehmigungen sind manuelle Prozesse, die durch den Filialleiter durchgeführt werden müssen.
Die Fahranfängerprüfung, die Verfügbarkeitsprüfung sowie die Preisberechnung lassen sich über
Geschäftsregeln abbilden. 
Im Folgenden beschreiben wir die Regeln, die wir aus den Aussagen der Aufgabenstellung
extrahiert haben.

Wir stellen die Regeln in folgendem Format vor:
\begin{itemize}
	\item Name bzw. kurze Beschreibung der Regel
	\item Kategorie, in der wir die Regel eingeordnet haben
	\item Definition der Regel in unserer DSL
\end{itemize}
Wir verzichten auf eine ausführliche Beschreibung der Regel in Deutsch, da unsere
DSL bereits eine präzise Beschreibung in Englisch darstellt. Da die Einteilung in die
Kategorien Definitions-, Ableitungs-, Prozess- und Einschränkungsregeln nicht immer
eindeutig ist, sind wir nach folgendem Schema vorgegangen:
\begin{description}
	\item[Definitionsregel:] Ein neuer Begriff wird eingeführt (z.~B. Fahranfänger)
	\item[Ableitungsregel:] Aus bestehenden Fakten werden neue Fakten abgeleitet (z.~B. Berechnung von Preisen)
	\item[Prozessregel:] Beschreiben das Vorgehen im Geschäftsprozess (z.~B. Genehmigung durch Filialleiter)
	\item[Einschränkungsregel:] Schränken ein, was möglich ist (z.~B. maximaler Rabatt)
\end{description}

\subsection{Fahranfängerprüfung}

Regel: Definition eines Fahranfängers \\
Kategorie: Definitionsregel

\begin{lstlisting}
rule "Is driving novice" 
	ruleflow-group "novice-check"
	when
		there is a request
		there is a customer with that request
		- age is less than 21 or has driving license for less than 2	
	then
		the customer is a driving novice	
end
\end{lstlisting}

Regel: Zuschlag für Fahranfänger \\
Kategorie: Prozessregel

\begin{lstlisting}
rule "Extra charge if at least one customer is a novice" 
	ruleflow-group "novice-check"
	when
		there is a request
		there exists a customer who is driving novice
	then
		set extra charge to 10%	
end
\end{lstlisting}

Regel: Genehmigung für höhere Klassen bei Fahranfängern \\
Kategorie: Prozessregel

\begin{lstlisting}
rule "Permission for higher classes required"
	ruleflow-group "novice-check"
	when
		there is a request
		- car class is not in ("Small", "Compact")
		there exists a customer who is driving novice
	then
		ask the boss for permission
end
\end{lstlisting}

\subsection{Verfügbarkeitsprüfung}

Regel: Downgrade aus Oberklasse, wenn nicht verfügbar \\
Kategorie: Prozessregel

\begin{lstlisting}
rule "Downgrade for upper class if not available"
	ruleflow-group "availibility-check"
	when
		there is a request
		- car class is "Upper"
		- car is not available
	then
		set car class to "Middle"
		set extra deduction to 10%
		update the request
end
\end{lstlisting}

Regel: Definition eines Upgrades für die Fahrzeugklasse \\
Kategorie: Definitionsregel

Es gibt eine Reihe von Regeln, die definieren welche Fahrzeugklassen ein Upgrade
für die angefragte Klasse darstellen. Folgende Regel ist ein Beispiel von
Klein- nach Kompaktwagen.

\begin{lstlisting}
rule "Upgrade from Small to Compact"
	ruleflow-group "availibility-check"
	when
		there is a request
		- car class is "Small"
		- "Compact" car is available
		there exists no upgrade
	then
		upgrade to "Compact" is possible	
end
\end{lstlisting}

Regel: Upgrade auf nächsthöhere Klasse, wenn nicht verfügbar \\
Kategorie: Prozessregel

\begin{lstlisting}
rule "If updgrade possible it requires permission"
	ruleflow-group "availibility-check"
	when
		there is a request
		- car is not available
		there is a possible upgrade
	then
		upgrade the car class
		mark the car for that request as available
end
\end{lstlisting}

\subsection{Preisberechnung}

Regel: Definition des Tagespreises \\
Kategorie: Definitionsregel

Es gibt eine Reihe von Regeln, die den Tagespreis für jede Fahrzeugklasse festlegen.
Folgende Regel ist ein Beispiel für Kleinwagen.

\begin{lstlisting}
rule "Daily price: Small class car" 
	ruleflow-group "price-calculation"
	when
		there is a request
		- car class is "Small"
		there is a rental day from that request
	then
		set the daily price to 40.00
end
\end{lstlisting}

Regel: Wochenenden und Feiertage haben reduzierten Tagespreis \\
Kategorie: Ableitungsregel

\begin{lstlisting}
rule "Daily price: Weekend discount"
	ruleflow-group "price-calculation"
	when
		there is a request
		there is a rental day from that request
		- is weekend or holiday
	then
		the daily price is discounted by 25%
end
\end{lstlisting}

\newpage
Regel: Jeder siebte Tag ist kostenlos \\
Kategorie: Ableitungsregel

\begin{lstlisting}
rule "Daily price: Free seventh day"
	ruleflow-group "price-calculation"
	when
		there is a request
		there is a rental day from that request
		- day index is in (7, 14, 21, 28)
	then
		set the daily price to 0.00
end
\end{lstlisting}

Regel: Basispreis ist die Summe der Tagespreise \\
Kategorie: Ableitungsregel

\begin{lstlisting}
rule "Base price is the sum of daily prices"
	ruleflow-group "price-calculation"
	when
		there is a request
		there is a rental day from that request
	then
		add the daily price to the base price
end
\end{lstlisting}

Regel: Neukundenrabatt \\
Kategorie: Ableitungsregel

\begin{lstlisting}
rule "Discount: All customers are new customers"
	ruleflow-group "price-calculation"
	when
		there is a request
		every customer from that request is a new customer
	then
		add 10.00 to the discount
end
\end{lstlisting}

\newpage
Regel: Reklamationsrabatt \\
Kategorie: Ableitungsregel

\begin{lstlisting}
rule "Discount: All customers have an old reclamation"
	ruleflow-group "price-calculation"
	when
		there is a request
		every customer from that request had a valid reclamation
	then
		discount the base price by 20%
end
\end{lstlisting}

Regel: Rabatt für Sicherheitstraining \\
Kategorie: Ableitungsregel

\begin{lstlisting}
rule "Discount: All customers have participated in security training"
	ruleflow-group "price-calculation"
	when
		there is a request
		every customer from that request had a security training
	then
		discount the base price by 5%;
end
\end{lstlisting}

Regel: Maximaler Rabatt \\
Kategorie: Einschränkungsregel

\begin{lstlisting}
rule "Discount is limited to 100 Euro"
	ruleflow-group "price-calculation"
	when
		there is a request
		- discount is greater than 100.00
	then
		set the discount to 100.00
end
\end{lstlisting}

\newpage
Regel: Endpreis berechnen \\
Kategorie: Ableitungsregel

\begin{lstlisting}
rule "Final Price: Calculation"
	ruleflow-group "price-calculation"
	when
		there is a request
	then
		set the final price to the base price
		subtract the discount from the final price
end
\end{lstlisting}

Regel: Automatikgebühr abziehen \\
Kategorie: Ableitungsregel

\begin{lstlisting}
rule "Final Price: Automatic fee"
	ruleflow-group "price-calculation"
	when
		there is a request
		- for an automatic car
	then
		add a 5% fee of the base price to the final price
end
\end{lstlisting}

Regel: Gesamtpreis berechnen \\
Kategorie: Ableitungsregel

\begin{lstlisting}
rule "Total Price"
	ruleflow-group "price-calculation"
	when
		there is a request
	then
		set percent to 100% plus extra charge minus extra deduction
		set the total price to percent% of the final price
end
\end{lstlisting}


\newpage
\section{Prototyp}

Die Umsetzung des Prototypen haben wir in sechs Schritte eingeteilt:
\begin{enumerate}
	\item Festlegung des Datenmodells
	\item Schreiben von Unit-Tests
	\item Definition des Geschäftsprozesses
	\item Implementierung der Regeln in normaler Regelsprache
	\item Transformation der Regeln in eine DSL
	\item Entwurf und Implementierung einer Oberfläche
\end{enumerate}

\subsection{Datenmodell}

Die zu modellierenden Daten umfassen Kunden- und Anfragedaten. Diese beiden Datensätze
werden als Java-Klassen modelliert (\texttt{Customer} und \texttt{RentalRequest}). Zusätzlich werden die
Informationen über die zur Verfügung stehenden Autos in der Klasse \texttt{Garage} gespeichert.
Die einzelnen Miettage werden als \texttt{RentalDay} modelliert. Das Klassendiagramm in
Abbildung \ref{fig:Class_All} zeigt alle Klassen.

\begin{figure}[tbh]
\centering
\includegraphics[width=1.0\linewidth]{Bilder/Class_All}
\caption{Klassendiagramm}
\label{fig:Class_All}
\end{figure}

Die Klasse \texttt{RentalRequest} dient sowohl der Eingabe als auch der Ausgabe von Daten.
Das heißt, dass Regeln und Prozesse ggf. schreibend auf die aktuelle Instanz zugreifen.

\subsection{Unit-Tests: Anwendungssituation}

Nachdem wir das Datenmodell festgelegt haben, konnten wir Unit-Tests für die 
Anwendungssituationen aus der Aufgabenstellung schreiben. Dabei ist uns aufgefallen,
dass bei sequentieller Abarbeitung der Kunden A bis G die Wagen der Mittel- und
Oberklasse schnell ausgehen. Dies führt dazu, dass die Anfragen der Kunden E und F
direkt abgelehnt werden. Für die Unit-Tests haben wir zusätzlich die Fälle betrachtet,
wenn es genügend Wagen in der angeforderten Klasse gibt.

Die erwarteten Preise in den einzelnen Anwendungssituationen haben wir manuell
mithilfe des initial erstellten Flussdiagramms ermittelt. Die Ergebnisse dieser
manuellen Auswertung befinden sich in Anhang \ref{sec:Anwendungssituation}.

In der Java-Klasse \texttt{SituationTest} werden die vorgegebenen Situationen getestet.
Zusätzlich haben wir in der Klasse \texttt{ExtraTest} versucht, bisher noch nicht
getestete Szenarien abzubilden. Dazu zählen z.~B. die Automatikgebühr und verschiedene
Genehmigungszenarien.

\subsection{Definition des Geschäftsprozesses}

Die Autovermietung ist ein Geschäftsprozess, in dem unterschiedliche Regelgruppen aktiviert
werden müssen. Außerdem ist abhängig von der Anfrage eine menschliche Interaktion durch den
Filialleiter in Form einer Genehmigung erforderlich. Dies haben wir mittels
jBPM als Prozess definiert.

Die Integration der Geschäftsregeln haben wir durch die Zuordnung zu Ruleflow-Groups
in der Regeldatei und das Einbinden von Business-Rule-Tasks in den Prozess erreicht.
Um die Genehmigung durch den Filialleiter zu ermöglichen, haben wir User- bzw.
Human-Tasks eingebunden.

Der gesamte Prozess wird in Abbildung \ref{fig:Process_BPMN} dargestellt.

\begin{figure}[p]
\centering
\includegraphics[width=0.8\linewidth]{Bilder/rental-process}
\caption{Geschäftsprozess in jBPM}
\label{fig:Process_BPMN}
\end{figure}

\subsection{Normale Regelsprache}

Um mögliche Fehlerquellen zu vermeiden, haben wir die Geschäftsregeln zuerst
in der normalen Rule-Language implementiert. Durch die ausführlichen Unit-Tests
konnten wir Fehler schnell finden und korrigieren. Die so erstellte Regelbasis
dient als Grundlage zur Erstellung der DSL. Anhang \ref{anh:normal-rules} enthält
die entsprechende Regeldatei.

\subsection{Transformation in DSL}

Wir haben uns entschieden unsere DSL in Englisch zu implementieren, um eine Mischung
von Englisch und Deutsch zu vermeiden. Bei der Definition haben wir uns an der
einfachen Regeldatei orientiert. Nachdem alle Regeln übersetzt waren, haben wir
die Unit-Tests auf die in der DSL definierten Regeln umgestellt. Dadurch haben wir
Fehler bei der Transformation schnell finden und beheben können. Die so erstellten
Regeln wurden bereits in Abschnitt \ref{sec:rules} vorgestellt.

\subsection{Oberfläche}
Die Screenshots der erstellten Oberfläche befinden sich in Anhang \ref{anh:screenshots} und werden
in der folgenden Beschreibung einzeln referenziert.

Zur einfachen Eingabe von Testfällen habe wir eine GUI in Swing programmiert.
Diese besteht aus einer Haupt-Eingabemaske (s. Abb. \ref{fig:Autovermietung}),
über die ein Mietgesuch eingegeben werden kann, einem Fahrer-Verwaltungsdialog  (s. Abb. \ref{fig:kunde})
und einem Garagen-Konfigurationsdialog (s. Abb. \ref{fig:Garage}).
Der Dialog zum Verwalten der Fahrer ermöglicht Eingaben über den Namen des Fahrers,
das Alter des Fahrers, wie lange der Fahrer schon einen Führerschein besitzt, 
ob er ein Neukunde ist, eine Reklamation vorweisen kann und ob er ein Sicherheitstraining absolviert hat.
Über den Garagen-Konfigurationsdialog kann die Anzahl der zu Verfügung stehenden Fahrzeuge in den einzelnen Fahrzeugklassen eingestellt werden. 
In der Haupt-Eingabemaske kann der Garagendialog aufgerufen werden und es können Fahrer über den Fahrerdialog hinzugefügt und verwaltet werden.
Alle Fahrer einer Mietanfrage werden im Hauptfenster angezeigt.
Das Hauptfenster bietet die Möglichkeit zum Einstellen des Mietstartdatums, der Mietdauer, der Fahrzeugklasse des Mietwagens und, ob der Mietwagen eine Automatikschaltung haben soll. 

Nachdem alle Eingaben gemacht wurden, kann eine Mietanfrage an das System gesendet werden. Abhängig von den Eingabewerten muss ein Fahranfänger  (s. Abb. \ref{fig:Fahränfanger}) oder ein Fahrzeugupgrade  (s. Abb. \ref{fig:Upgrade}) genehmigt werden. Wird eine Genehmigung nicht erteilt oder steht kein Fahrzeug mehr zur Verfügung, wird die Anfrage abgelehnt  (s. Abb. \ref{fig:ergebnisabgelehnt}).
Wurde die Anfrage angenommen, wird in einem abschließendem Dialog die Fahrzeugklasse und die zu zahlenden Kosten angezeigt  (s. Abb. \ref{fig:ergebnis}).

\newpage
\section{Fazit}
Am Anfang dieses Projektes haben wir zunächst die groben Strukturen aus der Aufgabenstellung extrahiert. Um daraus die Geschäftsregeln zu definieren, nutzten wir die dreizehn Aussagen. Die Regeln wurden von uns zunächst als normale Regelsprache implementiert. Der letzte Schritt bestand darin die Regelsprache in eine DSL zu überführen. Der zusätzliche Schritt über die normale Regelsprache half dabei Fehlerquellen zu vermeiden. Durch die Vorlesung und die ausführliche Onlinedokumentation gab es wenige Probleme bei der Ausführung der einzelnen Arbeitsschritte.

Es hat sich gezeigt, dass die Verwendung einer DSL für ein Projekt dieser Größe eine gute Alternative zur normalen Regelsprache ist. Sowohl Programmierer als auch Fachkundige ohne Programmiererfahrung können die DSL lesen und kleinere Änderungen machen. Das hilft besonders Personen ohne eine Affinität fürs Programmieren, da diese Fehler in der DSL finden und einem entsprechenden Programmierer melden können. Leider sind größere Änderungen an der DSL nicht ohne Programmieraufwand zu bewerkstelligen. Eine DSL muss trotz ihrer Umgangssprache noch eine feste Struktur aufweisen.

Die Integration von Drools und jBPM in Eclipse gefiel uns sehr gut. Dadurch war es einfach die Prozesse und Regeln zu implementieren. Des Weiteren war es schnell möglich, Unit-Tests zu definieren, um die Regeln zu testen. Der letzte Schritt war die Integration der Prozess-Engine in die grafische Benutzeroberfläche, wobei es keine nennenswerte Probleme gab.

Als endgültiges Fazit können wir sagen, dass das Arbeiten mit jBPM und Drools eine gute Erfahrung war, auch wenn einige Arbeitsschritte auf den ersten Blick eher komplex erscheinen. Bei kleineren Projekten halten wir eine reine Programmierlösung für den besseren Ansatz. Für größere Projekte mit umfangreichen Geschäftsregeln, die in Prozesse
integriert werden müssen, kann jBPM eine akzeptable Lösung sein. 

%\appendix
\newpage
\appendix

\section{Anwendungssituation}
\label{sec:Anwendungssituation}

Wir analysieren die Situation am 21. März 2016 in der Autovermietungsstation.
Dazu beschreiben wir zunächst die Ausgangssituation und dann die einzelnen Kundenanfragen.

\subsection{Ausgangssituation}

Am 21. März 2016 stehen zu jeder Fahrzeugklasse zwei Fahrzeuge zur Verfügung:

\begin{tabular}{|c|c|c|c|}
	\hline \textbf{Kleinklasse} & \textbf{Kompaktklasse} & \textbf{Mittelklasse} & \textbf{Oberklasse}  \\ 
	\hline 2 & 2 & 2 & 2 \\ 
	\hline 
\end{tabular} 

Relevante Feiertage:
\begin{itemize}
	\item Fr 25.03.2016: Karfreitag
	\item Mo 28.03.2016: Ostermontag
	\item So 01.05.2016: 1. Mai / Tag der Arbeit
\end{itemize}

Vorgehen:
\begin{itemize}
	\item Wir runden mathematisch, weil das einfacher zu implementieren ist.
	\item Nicht spezifizierte Anfrageparameter nehmen einen Standardwert an (Nein, bzw. 0).
	\item Bei nicht angegebenem Alter ist der Kunde älter als 21 Jahre.
	\item Der Filialleiter genehmigt alle Anfragen.
	\item Wenn die Anfrage abgelehnt wird, weil keine Wagen der angefragten Klasse vorhanden sind,
	wird als Alternative dieselbe Anfrage mit vorhandener Klasse betrachtet.
\end{itemize}

\subsection{Kunde A}

Kundendaten:\\
\begin{tabular}{|c|c|c|c|c|}
	\hline \textbf{Alter} & \textbf{Führerschein} & \textbf{Neukunde} & \textbf{Reklamation} & \textbf{Sicherheitstraining} \\ 
	\hline 42 & 20 & Nein & Nein & Ja \\ 
	\hline 
\end{tabular} 
\\\\
Anfrage:\\
\begin{tabular}{|c|c|c|}
	\hline \textbf{Klasse} & \textbf{Dauer in Tagen} & \textbf{Automatik} \\ 
	\hline Mittel & 5 & Nein \\ 
	\hline 
\end{tabular}

\vspace{12pt}
Preisberechnung:
\begin{itemize}
	\item Zeitraum: 4 Wochentage, 0 Wochenende, 1 Feiertag
	\item Fahranfängerprüfung: Bestanden, Zuschlag: 0\%
	\item Verfügbarkeitsprüfung: Bestanden, Nachlass: 0\%
	\item Tagespreis: 70€/Wochentag, 52,50€/Feiertag
	\item Basispreis = 70€ * 4 + 52,50€ * 1 = 332,50€
	\item Rabatt = 5\% * Basispreis = 16,63€
	\item Endpreis = 315,87€
	\item Gesamtpreis = Endpreis = 315,87€
\end{itemize}

Zustand nach Vermietung:\\
\begin{tabular}{|c|c|c|c|}
	\hline \textbf{Kleinklasse} & \textbf{Kompaktklasse} & \textbf{Mittelklasse} & \textbf{Oberklasse}  \\ 
	\hline 2 & 2 & 1 & 2 \\ 
	\hline 
\end{tabular} 

\subsection{Kunde B}

Kundendaten:\\
\begin{tabular}{|c|c|c|c|c|}
	\hline \textbf{Alter} & \textbf{Führerschein} & \textbf{Neukunde} & \textbf{Reklamation} & \textbf{Sicherheitstraining} \\ 
	\hline 20 & 3 & Ja & Nein & Nein \\ 
	\hline 
\end{tabular} 
\\\\
Anfrage:\\
\begin{tabular}{|c|c|c|}
	\hline \textbf{Klasse} & \textbf{Dauer in Tagen} & \textbf{Automatik} \\ 
	\hline Mittel & 1 & Nein \\ 
	\hline 
\end{tabular}

Preisberechnung:
\begin{itemize}
	\item Zeitraum: 1 Wochentag, 0 Wochenende oder Feiertage
	\item Fahranfängerprüfung: Genehmigung, Bestanden, Zuschlag: 10\%
	\item Verfügbarkeitsprüfung: Bestanden, Nachlass: 0\%
	\item Tagespreis: 70€/Wochentag, 52,50€/Feiertag
	\item Basispreis = 70€ * 1 + 52,50€ * 0 = 70,00€
	\item Rabatt = 10,00€
	\item Endpreis = 60,00€
	\item Gesamtpreis = 110\% * Endpreis = 66,00€
\end{itemize}

Zustand nach Vermietung:\\
\begin{tabular}{|c|c|c|c|}
	\hline \textbf{Kleinklasse} & \textbf{Kompaktklasse} & \textbf{Mittelklasse} & \textbf{Oberklasse}  \\ 
	\hline 2 & 2 & 0 & 2 \\ 
	\hline 
\end{tabular} 

\subsection{Kunde C}

Kundendaten:\\
\begin{tabular}{|c|c|c|c|c|}
	\hline \textbf{Alter} & \textbf{Führerschein} & \textbf{Neukunde} & \textbf{Reklamation} & \textbf{Sicherheitstraining} \\ 
	\hline 30 & 12 & Nein & Nein & Nein \\ 
	\hline 
\end{tabular} 
\\\\
Anfrage:\\
\begin{tabular}{|c|c|c|}
	\hline \textbf{Klasse} & \textbf{Dauer in Tagen} & \textbf{Automatik} \\ 
	\hline Mittel & 7 & Nein \\ 
	\hline 
\end{tabular}

Zwischenergebnis:
\begin{itemize}
	\item Zeitraum: 4 Wochentag, 2 Wochenende oder Feiertage, 1 Gratistag
	\item Fahranfängerprüfung: Bestanden, Zuschlag: 0\%
	\item Verfügbarkeitsprüfung: Genehmigung, Bestanden, Upgrade auf höhere Klasse, Nachlass: 0\%
	\item Tagespreis: 70€/Wochentag, 52,50€/Feiertag
	\item Basispreis = 70€ * 4 + 52,50€ * 2 + 0€ * 1 = 385,00€
	\item Rabatt = 0,00€
	\item Endpreis = 385,00€
	\item Gesamtpreis = 100\% * Endpreis = 385,00€
\end{itemize}

Zustand nach Vermietung:\\
\begin{tabular}{|c|c|c|c|}
	\hline \textbf{Kleinklasse} & \textbf{Kompaktklasse} & \textbf{Mittelklasse} & \textbf{Oberklasse}  \\ 
	\hline 2 & 2 & 0 & 1 \\ 
	\hline 
\end{tabular} 

\subsection{Kunde D}

Kundendaten:\\
\begin{tabular}{|c|c|c|c|c|}
	\hline \textbf{Alter} & \textbf{Führerschein} & \textbf{Neukunde} & \textbf{Reklamation} & \textbf{Sicherheitstraining} \\ 
	\hline 35 & 10 & Ja & Nein & Nein \\ 
	\hline 
\end{tabular} 
\\\\
Anfrage:\\
\begin{tabular}{|c|c|c|}
	\hline \textbf{Klasse} & \textbf{Dauer in Tagen} & \textbf{Automatik} \\ 
	\hline Mittel & 1 & Nein \\ 
	\hline 
\end{tabular}

Preisberechnung:
\begin{itemize}
	\item Zeitraum: 1 Wochentag, 0 Wochenende oder Feiertage
	\item Fahranfängerprüfung: Bestanden, Zuschlag: 0\%
	\item Verfügbarkeitsprüfung: Genehmigung, Bestanden, Upgrade auf höhere Klasse, Nachlass: 0\%
	\item Tagespreis: 70€/Wochentag, 52,50€/Feiertag
	\item Basispreis = 70€ * 1 + 52,50€ * 0 = 70,00€
	\item Rabatt = 10,00€
	\item Endpreis = 60,00€
	\item Gesamtpreis = 100\% * Endpreis = 60,00€
\end{itemize}

Zustand nach Vermietung:\\
\begin{tabular}{|c|c|c|c|}
	\hline \textbf{Kleinklasse} & \textbf{Kompaktklasse} & \textbf{Mittelklasse} & \textbf{Oberklasse}  \\ 
	\hline 2 & 2 & 0 & 0 \\ 
	\hline 
\end{tabular}

\subsection{Kunde E (abgelehnt)}

Kundendaten:\\
\begin{tabular}{|c|c|c|c|c|}
	\hline \textbf{Alter} & \textbf{Führerschein} & \textbf{Neukunde} & \textbf{Reklamation} & \textbf{Sicherheitstraining} \\ 
	\hline 46 & 20 & Ja & Nein & Nein \\ 
	\hline 
\end{tabular} 
\\\\
Anfrage:\\
\begin{tabular}{|c|c|c|}
	\hline \textbf{Klasse} & \textbf{Dauer in Tagen} & \textbf{Automatik} \\ 
	\hline Ober & 6 & Nein \\ 
	\hline 
\end{tabular}

Preisberechnung:
\begin{itemize}
	\item Zeitraum: 4 Wochentag, 2 Wochenende oder Feiertage
	\item Fahranfängerprüfung: Bestanden, Zuschlag: 0\%
	\item Verfügbarkeitsprüfung: Genehmigung, Abgelehnt, Nachlass: 10\%, Klasse = Mittel
	
	% Es gibt noch eine Oberklasse:
	%\item Tagespreis: 90€/Wochentag, 67,50€/Feiertag
	%\item Basispreis = $90€ * 4 + 67,50€ * 2 = 495,00€$
	%	\item Rabatt = 10,00€
	%	\item Endpreis = 485,00€
	%	\item Gesamtpreis = 100\% * Endpreis = 485,00€
	
	% Es gibt keine Oberklasse aber eine Mittelklasse
	%	\item Tagespreis: 70€/Wochentag, 52,50€/Feiertag
	%	\item Basispreis = $70€ * 4 + 52,50€ * 2 = 385,00€$
	%	\item Rabatt = 10,00€
	%	\item Endpreis = 375,00€
	%	\item Gesamtpreis = 90\% * Endpreis = 337,50€
\end{itemize}

Abgelehnt: Keine Änderung
Zustand nach Vermietung:\\
\begin{tabular}{|c|c|c|c|}
	\hline \textbf{Kleinklasse} & \textbf{Kompaktklasse} & \textbf{Mittelklasse} & \textbf{Oberklasse}  \\ 
	\hline 2 & 2 & 0 & 0 \\ 
	\hline 
\end{tabular}

\subsection{Kunde F (abgelehnt)}

Kundendaten:\\
\begin{tabular}{|c|c|c|c|c|}
	\hline \textbf{Alter} & \textbf{Führerschein} & \textbf{Neukunde} & \textbf{Reklamation} & \textbf{Sicherheitstraining} \\ 
	\hline 32 & 10 & Nein & Nein & Ja \\ 
	\hline 
\end{tabular} 
\\\\
Anfrage:\\
\begin{tabular}{|c|c|c|}
	\hline \textbf{Klasse} & \textbf{Dauer in Tagen} & \textbf{Automatik} \\ 
	\hline Ober & 4 & Nein \\ 
	\hline 
\end{tabular}

Zwischenergebnis:
\begin{itemize}
	\item Zeitraum: 4 Wochentag, 0 Wochenende oder Feiertage
	\item Fahranfängerprüfung: Bestanden, Zuschlag: 0\%
	\item Verfügbarkeitsprüfung: Genehmigung, Abgelehnt, Nachlass: 0\%
	%	\item Tagespreis: 90€/Wochentag, 67,50€/Feiertag
	%	\item Basispreis = $90€ * 4 + 67,50€ * 0 = 360,00€$
	%	\item Rabatt = 20\% * 360€ + 5\% * 360€ = 72€ + 18€ = 90€
	%	\item Endpreis = 270,00€
	%	\item Gesamtpreis = 100\% * Endpreis = 270,00€
	
	% Siehe E
\end{itemize}

Abgelehnt: Keine Änderung
Zustand nach Vermietung:\\
\begin{tabular}{|c|c|c|c|}
	\hline \textbf{Kleinklasse} & \textbf{Kompaktklasse} & \textbf{Mittelklasse} & \textbf{Oberklasse}  \\ 
	\hline 2 & 2 & 0 & 0 \\ 
	\hline 
\end{tabular}

\subsection{Kunde G}

Kundendaten:\\
\begin{tabular}{|c|c|c|c|c|}
	\hline \textbf{Alter} & \textbf{Führerschein} & \textbf{Neukunde} & \textbf{Reklamation} & \textbf{Sicherheitstraining} \\ 
	\hline 30 & 6 & Nein & Nein & Ja \\ 
	\hline 25 & 1 & Nein & Nein & Nein \\ 
	\hline 
\end{tabular} 
\\\\
Anfrage:\\
\begin{tabular}{|c|c|c|}
	\hline \textbf{Klasse} & \textbf{Dauer in Tagen} & \textbf{Automatik} \\ 
	\hline Kompakt & 1 & Nein \\ 
	\hline 
\end{tabular}

Zwischenergebnis:
\begin{itemize}
	\item Zeitraum: 1 Wochentag, 0 Wochenende oder Feiertage
	\item Fahranfängerprüfung: Bestanden, Zuschlag: 10\%
	\item Verfügbarkeitsprüfung: Bestanden, Nachlass: 0\%
	\item Tagespreis: 50€/Wochentag, 37,50€/Feiertag
	\item Basispreis = 50€ * 1 + 37,50€ * 0 = 50,00€
	\item Rabatt = 0€
	\item Endpreis = 50,00€
	\item Gesamtpreis = 110\% * Endpreis = 55,00€
\end{itemize}

Zustand nach Vermietung:\\
\begin{tabular}{|c|c|c|c|}
	\hline \textbf{Kleinklasse} & \textbf{Kompaktklasse} & \textbf{Mittelklasse} & \textbf{Oberklasse}  \\ 
	\hline 2 & 2 & 0 & 0 \\ 
	\hline 
\end{tabular}

\subsection{Kunde E (angenommen)}

Wir nehmen an, es gäbe ausreichend Wagen jeder Klasse:

\begin{tabular}{|c|c|c|c|}
	\hline \textbf{Kleinklasse} & \textbf{Kompaktklasse} & \textbf{Mittelklasse} & \textbf{Oberklasse}  \\ 
	\hline 2 & 2 & 2 & 2 \\ 
	\hline 
\end{tabular}

Kundendaten:\\
\begin{tabular}{|c|c|c|c|c|}
	\hline \textbf{Alter} & \textbf{Führerschein} & \textbf{Neukunde} & \textbf{Reklamation} & \textbf{Sicherheitstraining} \\ 
	\hline 46 & 20 & Ja & Nein & Nein \\ 
	\hline 
\end{tabular} 
\\\\
Anfrage:\\
\begin{tabular}{|c|c|c|}
	\hline \textbf{Klasse} & \textbf{Dauer in Tagen} & \textbf{Automatik} \\ 
	\hline Ober & 6 & Nein \\ 
	\hline 
\end{tabular}

Preisberechnung:
\begin{itemize}
	\item Zeitraum: 4 Wochentag, 2 Wochenende oder Feiertage
	\item Fahranfängerprüfung: Bestanden, Zuschlag: 0\%
	\item Verfügbarkeitsprüfung: Bestanden, Nachlass: 0\%
	
	% Es gibt noch eine Oberklasse:
	\item Tagespreis: 90€/Wochentag, 67,50€/Feiertag
	\item Basispreis = $90€ * 4 + 67,50€ * 2 = 495,00€$
	\item Rabatt = 10,00€
	\item Endpreis = 485,00€
	\item Gesamtpreis = 100\% * Endpreis = 485,00€
	
	% Es gibt keine Oberklasse aber eine Mittelklasse
	%	\item Tagespreis: 70€/Wochentag, 52,50€/Feiertag
	%	\item Basispreis = $70€ * 4 + 52,50€ * 2 = 385,00€$
	%	\item Rabatt = 10,00€
	%	\item Endpreis = 375,00€
	%	\item Gesamtpreis = 90\% * Endpreis = 337,50€
\end{itemize}

Zustand nach Vermietung:\\
\begin{tabular}{|c|c|c|c|}
	\hline \textbf{Kleinklasse} & \textbf{Kompaktklasse} & \textbf{Mittelklasse} & \textbf{Oberklasse}  \\ 
	\hline 2 & 2 & 2 & 1 \\ 
	\hline 
\end{tabular}

\subsection{Kunde F (angenommen)}

Wir nehmen an, es gäbe ausreichend Wagen jeder Klasse:

\begin{tabular}{|c|c|c|c|}
	\hline \textbf{Kleinklasse} & \textbf{Kompaktklasse} & \textbf{Mittelklasse} & \textbf{Oberklasse}  \\ 
	\hline 2 & 2 & 2 & 2 \\ 
	\hline 
\end{tabular}

Kundendaten:\\
\begin{tabular}{|c|c|c|c|c|}
	\hline \textbf{Alter} & \textbf{Führerschein} & \textbf{Neukunde} & \textbf{Reklamation} & \textbf{Sicherheitstraining} \\ 
	\hline 32 & 10 & Nein & Nein & Ja \\ 
	\hline 
\end{tabular} 
\\\\
Anfrage:\\
\begin{tabular}{|c|c|c|}
	\hline \textbf{Klasse} & \textbf{Dauer in Tagen} & \textbf{Automatik} \\ 
	\hline Ober & 4 & Nein \\ 
	\hline 
\end{tabular}

Zwischenergebnis:
\begin{itemize}
	\item Zeitraum: 4 Wochentag, 0 Wochenende oder Feiertage
	\item Fahranfängerprüfung: Bestanden, Zuschlag: 0\%
	\item Verfügbarkeitsprüfung: Bestanden, Nachlass: 0\%
	\item Tagespreis: 90€/Wochentag, 67,50€/Feiertag
	\item Basispreis = 90€ * 4 + 67,50€ * 0 = 360,00€
	\item Rabatt = 20\% * 360€ + 5\% * 360€ = 72€ + 18€ = 90€
	\item Endpreis = 270,00€
	\item Gesamtpreis = 100\% * Endpreis = 270,00€
	
	% Siehe E
\end{itemize}

Abgelehnt: Keine Änderung
Zustand nach Vermietung:\\
\begin{tabular}{|c|c|c|c|}
	\hline \textbf{Kleinklasse} & \textbf{Kompaktklasse} & \textbf{Mittelklasse} & \textbf{Oberklasse}  \\ 
	\hline 2 & 2 & 2 & 1 \\ 
	\hline 
\end{tabular}

\newpage
\section{Normale Rule-Language}
\label{anh:normal-rules}

\lstinputlisting{src/rental-rules.drl}

\newpage
\section{Screenshots der GUI}
\label{anh:screenshots}

\begin{figure}[htb]
	\centering
	\includegraphics[width=0.53\linewidth]{Bilder/Screenshots/Autovermietung}
	\caption{Haupt-Eingabemaske }
	\label{fig:Autovermietung}
\end{figure}


\begin{figure}
	\centering
	\includegraphics[width=0.7\linewidth]{Bilder/Screenshots/kunde}
	\caption{Kundendialog}
	\label{fig:kunde}
\end{figure}

\begin{figure}
	\centering
	\includegraphics[width=0.7\linewidth]{Bilder/Screenshots/Garage}
	\caption{Garagendialog}
	\label{fig:Garage}
\end{figure}

\begin{figure}
	\centering
	\includegraphics[width=0.7\linewidth]{Bilder/Screenshots/Fahranfanger}
	\caption{Fahranfänger genehmigen}
	\label{fig:Fahränfanger}
\end{figure}

\begin{figure}
	\centering
	\includegraphics[width=1.0\linewidth]{Bilder/Screenshots/Upgrade}
	\caption{Fahrzeugklassen-Upgrade genehmigen}
	\label{fig:Upgrade}
\end{figure}

\begin{figure}
	\centering
	\includegraphics[width=0.7\linewidth]{Bilder/Screenshots/ergebnis-abgelehnt}
	\caption{Anfrage abgelehnt}
	\label{fig:ergebnisabgelehnt}
\end{figure}

\begin{figure}
	\centering
	\includegraphics[width=0.7\linewidth]{Bilder/Screenshots/ergebnis}
	\caption{Ergebnis einer genehmigten Anfrage}
	\label{fig:ergebnis}
\end{figure}

\newpage
\section{Aufgabenstellung: Autovermietung – Domain Specific Language}
\label{anh:Aufgabe}

Der folgende Text wurde wörtlich aus der Aufgabenstellung für dieses Projekt übernommen.

\subsection*{Aussagen}

\begin{enumerate}
	\item Die Autovermietung bietet folgende Fahrzeugklassen zu den genannten Tagesmietpreisen an:
	\begin{enumerate}
		\item Kleinwagen, 40 Euro/Tag
		\item Kompaktklasse, 50 Euro/Tag
		\item Mittelkasse, 70 Euro/Tag
		\item Oberklasse, 90 Euro/Tag
	\end{enumerate}
	\item Die Tagesmietpreise reduzieren sich an Wochenendtagen (Samstag und Sonntag) sowie an
	bundesweiten Feiertagen um 25\%.
	\item Die Mietpreise sind zeitlich gestaffelt. Bei einer Mietdauer von 1-6 Tagen ist pro Tag 100\% des
	Tagesmietpreises zu zahlen. Bei einer Mietdauer 7 Tagen ist der siebte Tag frei, entsprechendes
	gilt für den 14, 21 und 28 Tag. Eine Vermietung für mehr als 28 Tage ist nicht möglich.
	\item Neukunden erhalten bei der ersten Vermietung einen einmaligen Rabatt von 10 Euro
	(unabhängig von der Mietdauer).
	\item Mieter, die eine berechtigte Reklamation vorbringen, erhalten bei der nächsten Buchung einen
	Rabatt von 20\%, maximal jedoch 100 Euro.
	\item Fahrer, die ein Sicherheitstraining absolviert haben, erhalten einen Rabatt von 5\%.
	\item Wird ein Fahrzeug von mehreren Fahrern genutzt, so gelten die jeweils schlechteren Konditionen
	(also wird z.B. kein Rabatt für ein Sicherheitstraining gewährt, wenn nur einer der beiden Fahrer
	ein Sicherheitstraining absolviert hat).
	\item Wird ein Fahrzeug mit Automatik verlangt, erhöht sich der Mietpreis um 5\%.
	\item Der Gesamtrabatt kann unabhängig von der Mietdauer 100 Euro nicht überschreiten. Gratistage
	und reduzierte Preise an Wochenenden und Feiertagen zählen nicht als Rabatt.
	\item Fahranfänger, die erst seit weniger als 2 Jahren einen Führerschein besitzen oder die jünger als
	21 Jahre sind, zahlen einen Aufschlag von 10\% auf alle Mietendpreise.
	\item Fahranfänger, wie zuvor beschrieben, können nur einen Kleinwagen oder ein Modell aus der
	Kompaktklasse mieten. . Nach einer expliziten Genehmigung durch den Filialleiter können im
	Einzelfall jedoch auch Fahrzeuge aus anderen Klassen gemietet werden.
	\item Bucht ein Kunde ein Auto aus einer Klasse, in der kein Wagen mehr zur Verfügung steht, so erhält
	er nach Genehmigung durch den Filialleiter ein kostenloses Upgrade für die nächsthöhere Klasse.
	Steht auch dort kein Fahrzeug zur Verfügung oder lehnt der Filialleiter das Upgrade ab, so ist das
	Mietgesuch abzulehnen.
	\item Möchte ein Kunde ein Oberklasse-Auto buchen und es steht kein Fahrzeug mehr zur Verfügung,
	so erhält er ein Fahrzeug der Mittelklasse mit einem zusätzlichen Nachlass auf den Mietendpreis
	von 10\%. Regel 9 ist dabei nicht anzuwenden. Ist kein Mittelklasse-Fahrzeug verfügbar, so ist das
	Mietgesuch abzulehnen.
\end{enumerate}

Extrahieren Sie die Regeln und mögliche Workflows aus den vorgenannten 13 Aussagen. Erstellen Sie
ein Flussdiagramm, das den Preisfindungsprozess zeigt. Falls sich Regeln widersprechen oder Regeln
fehlen, ergänzen oder verändern Sie das Beispiel sinnvoll.

Ordnen Sie die Regeln in die in der Veranstaltung genannten Regelklassen ein.


Bauen Sie den Prototyp einer Anwendung, die Mietpreis berechnet und die gemietete Wagenklasse
ausgibt, wenn gewünschte Fahrzeugklasse, Mietdauer, Mietbeginn und andere erforderliche Daten
eingegeben werden. [Sie können vereinfachend davon ausgehen, dass keine Datenbank verwendet
wird, sondern alle Kundendaten immer eingegeben werden.] Der Prototyp darf eine einfache
textbasierte Benutzungsoberfläche (Konsole) haben. \textbf{Die Regeln sollen unter Verwendung einer
Domain Specific Language (DSL) in Drools abgebildet werden. Für die eventuelle Workflows verwenden Sie bitte jBPM.}

\subsection*{Anwendungsituation}
Am 21. März 2016 stehen in Ihrer Autovermietungsstation jeweils 2 Fahrzeuge jeder Fahrzeugklasse
bereit. Nacheinander kommen die folgenden Kunden. Ihr Programm soll jeweils angeben, aus
welcher Klasse der Kunde ein Fahrzeug bekommt und welchen Gesamtpreis er bezahlen muss.

Spielen Sie die Situationen zunächst manuell durch, um die Ergebnisse für das Testszenario zu
erhalten. Erstellen Sie anschließend einen automatisierten Test.

\begin{enumerate}[\alph{enumi}]
\item Kunde A möchte (Umsatz laufendes Jahr 3000 Euro) ein Mittelkasse-Modell für einen
Zeitraum von 5 Tagen mieten. Der Fahrer hat ein Sicherheitstraining absolviert und besitzt
seit 20 Jahren einen Führerschein.
\item Der 20-jährige Neukunde B, der seit drei Jahren einen Führerschein besitzt, will einen
Mittelklasse-Wagen für einen Zeitraum von einem Tag mieten.
\item Kunde C (30 Jahre, Führerschein seit 12 Jahren, kein Neukunde) möchte ein MittelklasseModell
für 7 Tage mieten.
\item Kunde D (35 Jahre, Führerschein seit 10 Jahren, Neukunde) möchte ein Mittelklasse-Modell
für einen Tag mieten.
\item Ein 46-jähriger Neukunde E (Führerschein seit 20 Jahren) will einen Oberklasse-Wagen für
einen Zeitraum von 6 Tagen mieten.
\item Kunde F (32, der seit 10 Jahren einen Führerschein besitzt und ein Fahrsicherheitstraining
absolviert hat), hat bei der letzten Vermietung eine berechtige Beschwerde vorgebracht und
möchte einen Oberklasse-Wagen für 4 Tage mieten.
\item Kunde G, der das letzte Mal in 2010 ein Fahrzeug gemietet hat und ein Sicherheitstraining
absolviert hat, möchte einen Kompaktklasse-Wagen für einen Tag mieten. Seine Frau
(Führerscheinneuling) soll ebenfalls als Fahrerin eingetragen werden.
\end{enumerate}

\textbf{Anmerkung:} Sie können bei der Gestaltung der Dialog-Anwendung mit anderen Gruppen, die
ebenfalls eine Dialoganwendung benötigen, zusammenarbeiten.
%\newpage
%\printbibliography

\end{document}